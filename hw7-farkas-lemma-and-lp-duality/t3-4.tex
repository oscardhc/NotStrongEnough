\documentclass[UTF8, a4paper, linespread=1.5]{article}

\usepackage{tcolorbox, listings, algorithm, minted, algpseudocode}
\usepackage{geometry, savesym, amsmath, enumerate, indentfirst, color, amsthm, bm, extarrows, ulem}
\usepackage{amssymb}
\usepackage{nameref, hyperref}
 \geometry{top=3cm, bottom=3cm, left=1.5cm, right=1.5cm}

\usepackage{enumitem}
\setenumerate[1]{itemsep=0pt,partopsep=0pt,parsep=\parskip,topsep=5pt}
\setitemize[1]{itemsep=0pt,partopsep=0pt,parsep=\parskip,topsep=5pt}

\renewcommand\contentsname{Contents}

\tcbuselibrary{skins, breakable, theorems}

% \setlength{\leftskip}{10pt}
\setlength{\parindent}{10pt}
% \setlength{\parskip}{2em}
\renewcommand{\baselinestretch}{1.3}

\newcounter{RomanNumber}
\newcommand{\mrm}[1]{(\setcounter{RomanNumber}{#1}\Roman{RomanNumber})}

\newtcbtheorem{thm}{}
  {enhanced, theorem name and number, code={\edef\@currentlabelname{#2}}, 
  frame code={
        % \path[thick, draw] (frame.north west) -| (frame.north east) -| (frame.south east) -| (frame.south west) -| (frame.north west);
        \path[thick, draw] (frame.north west)  +(.5\baselineskip,0) -| +(0,-.5\baselineskip);
        % \path[thick, draw] (frame.north east) +(-.5\baselineskip,0) -| +(0,-.5\baselineskip);
        % \path[thick, draw] (frame.south west) +(.5\baselineskip,0) -| +(0,.5\baselineskip);
        \path[thick, draw] (frame.south east) +(-.5\baselineskip,0) -| +(0,.5\baselineskip);
    },
    left=1mm, right=1mm, top=1mm, bottom=1mm,
    colback=black!5,
    colframe=red!75!black,
    colbacktitle=black!0,
    coltitle=black!100,
    fonttitle=\bfseries}{thm}


\usepackage{environ}
\RenewEnviron{math}{%
\begin{align*}
\BODY
\end{align*}
}

\title{CS217 -- Algorithm Design and Analysis \\ Homework 7}
\date{\today}
\author{Not Strong Enough}


\begin{document}
    \begin{thm}{}{}
        Write down the dual of MCF. 
    \end{thm}

    \begin{proof}
        First, we write down MCF in terms of inequality: 
        \begin{align*}
            {\rm minimize}\ &\sum_{e\in E}c(e)f(e)\\
            {\rm subject\ to}\ &\sum_{e\in E:e=(u,t)}f(e)\geq 1 & \text{(corresponds to $y_t^+$ in dual form)}\\
                &\sum_{e\in E:e=(u,t)}-f(e)\geq -1 & \text{(corresponds to $y_t^-$ in dual form)}\\
                &\sum_{e\in E:e=(u,v)}f(e)-\sum_{e\in E:e=(v,w)}f(e)\geq 0,\forall v\in V\setminus\{s,t\} & \text{(corresponds to $y_v^+$ in dual form)}\\
                &\sum_{e\in E:e=(v,w)}f(e)-\sum_{e\in E:e=(u,v)}f(e)\geq 0,\forall v\in V\setminus\{s,t\} & \text{(corresponds to $y_v^-$ in dual form)}\\
                &f(e) \geq 0, \forall e \in E.
        \end{align*}
        Then write down its dual LP: (Let $S=\{e\in E:e=(s,v),v\in V\}$)
        \begin{align*}
            {\rm maximize}\ &y_t^+-y_t^-\\
            {\rm subject\ to}\ &(y_v^+-y_v^-)-(y_u^+-y_u^-)\leq c(e),\forall e=(u,v)\in E\setminus S\\
                        &y_v^+-y_v^-\leq c(e),\forall e=(s,v)\in S \\
                        & y_v^+, y_v^- \geq 0, \forall v\in V\setminus\{s,t\} \\
                        & y_t^+, y_t^- \geq 0.
        \end{align*}
        Now let $z_u=y_u^+-y_u^-$ for all vertices $u$, it turns to be: 
        \begin{align*}
            {\rm maximize}\ &z_t\\
            {\rm subject\ to}\ &z_v-z_u\leq c(e),\forall e=(u,v)\in E\setminus S\\
                        &z_v\leq c(e),\forall e=(s,v)\in S \\
                        & z_v \in \mathbf{R}, \forall v v\in V\setminus\{s,t\} \\
                        & z_t \in \mathbf{R}.
        \end{align*}
        Let $z_v'=z_v+z_s'$, for all vertices $u$ except for $s$ and the problem turns to: 
        \begin{align*}
            {\rm maximize}\ &z_t'-z_s'\\
            {\rm subject\ to}\ &z_v'-z_u'\leq c(e),\forall e=(u,v)\in E \\
            & z_v' \in \mathbf{R}, \forall v \in V.
        \end{align*}
    \end{proof}

    \begin{thm}{}{}
        Interpret the dual. Show that it is the LP formulation of a ``natural'' maximization problem on G
    \end{thm}

    \begin{proof}[Solution]
        Consider that each vertex has a potential, and for each edge $e=(u,v)$, 
        the potential of the terminal vertex $v$ is no greater than the potential of the start vertex $u$ plus $c(u, v)$, 
        and our goal is to maximize the potential of $t$. 
    \end{proof}
\end{document}
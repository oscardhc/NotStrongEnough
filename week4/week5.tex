\documentclass[UTF8, a4paper, linespread=1.5]{article}

\usepackage{tcolorbox, listings, algorithm, minted, algpseudocode}
\usepackage{geometry, savesym, amsmath, enumerate, indentfirst, color, amsthm, bm, extarrows, ulem}
\usepackage{amssymb}
\usepackage{nameref, hyperref}
 \geometry{top=3cm, bottom=3cm, left=1.5cm, right=1.5cm}

\usepackage{enumitem}
\setenumerate[1]{itemsep=0pt,partopsep=0pt,parsep=\parskip,topsep=5pt}
\setitemize[1]{itemsep=0pt,partopsep=0pt,parsep=\parskip,topsep=5pt}

\renewcommand\contentsname{Contents}

\tcbuselibrary{skins, breakable, theorems}

% \setlength{\leftskip}{10pt}
\setlength{\parindent}{10pt}
% \setlength{\parskip}{2em}
\renewcommand{\baselinestretch}{1.3}

\newcounter{RomanNumber}
\newcommand{\mrm}[1]{(\setcounter{RomanNumber}{#1}\Roman{RomanNumber})}

\newtcbtheorem{thm}{}
  {enhanced, theorem name and number, code={\edef\@currentlabelname{#2}}, 
  frame code={
        % \path[thick, draw] (frame.north west) -| (frame.north east) -| (frame.south east) -| (frame.south west) -| (frame.north west);
        \path[thick, draw] (frame.north west)  +(.5\baselineskip,0) -| +(0,-.5\baselineskip);
        % \path[thick, draw] (frame.north east) +(-.5\baselineskip,0) -| +(0,-.5\baselineskip);
        % \path[thick, draw] (frame.south west) +(.5\baselineskip,0) -| +(0,.5\baselineskip);
        \path[thick, draw] (frame.south east) +(-.5\baselineskip,0) -| +(0,.5\baselineskip);
    },
    left=1mm, right=1mm, top=1mm, bottom=1mm,
    colback=black!5,
    colframe=red!75!black,
    colbacktitle=black!0,
    coltitle=black!100,
    fonttitle=\bfseries}{thm}


\usepackage{environ}
\RenewEnviron{math}{%
\begin{align*}
\BODY
\end{align*}
}

\title{CS217 -- Algorithm Design and Analysis \\ Homework 7}
\date{\today}
\author{Not Strong Enough}



\begin{document}

\maketitle

\begin{thm}{}{}
    Let $T$ be a minimum spanning tree of $G$, and let $c\in \mathbb{R}$.
    Show that $T_c$ and $G_c$ have exactly the same connected components. (That is, two vertices $u,v\in V$ are connected in $T_c$ if and only if they are connected in $G_c$).
    You are encouraged to draw pictures to illustrate your proof.
\end{thm}
\begin{proof}[Solution]
    Since $T_c$ is a subset of $G_c$, if $u,v\in V$ are connected in $T_c$, then $u,v$ are connected in $G_c$.

    Now let's assume $u,v\in V$ are not connected in $T_c$, but are connected in $G_c$.
    We call the edge $(u,v)$ $x$, then $w(x)\leq c$.
    Also, the path in $T$ from $u$ to $v$ contains an edge whose weight is greater than c.
    We call it $y$, then $w(y)>  c$.
    If we add $x$ into $T$, this forms a cycle which contains both $x$ and $y$.
    After that, removing $y$ from $T$ still keeps the connectivity.
    We name $T'=T+x-y$.
    Now $T'$ becomes a tree, because it has $|V|-1$ edges and it's connected.
    The total weight of  $T'$ is $w(T)+w(x)-w(y)<w(T)$, which violates that $T$ is an MST\@.
    So $u,v$ are not connected in $G_c$ if they are not connected in $T_c$.

    In conclusion, two vertices $u,v\in V$ are connected in $T_c$ if and only if they are connected in $G_c$.
\end{proof}
\begin{thm}{}{}
    Suppose G is connected, and no two edges of G have the same weight. Show that G has exactly one minimum spanning tree
\end{thm}
\begin{proof}[Proof]
    Suppose there are two different minimum spanning trees, named $T$ and $T'$. 

    Now we remove an edge $e$ in $T$, and the graph becomes two connected parts. 
    If in $T'$ the two parts are connected by $e$ too, 
    then remove another edge in $T$ until the one in $T'$ is different from the one in $T$. 

    There should be such a pair: otherwise, each edge in $T$ is the same as that in $T'$, which is a contradiction. 

    Now consider such pair. Without loss of generality, assume that $w(e)<w(e')$. 
    Then if we replace $e'$ in $T'$ by $e$, the total weight of $T'$ is less, so $T'$ is not a minimum spanning tree. 
\end{proof}

\begin{thm}{}{}
    Suppose you have a polynomial-time algorithm that, given a multigraph $H$, computes the number of spanning trees of $H$. Using this algorithm as a subroutine, design a polynomial-time algorithm that, given a weighted graph $G$, computes the number of minimum spanning trees of $G$.
\end{thm}

\begin{algorithm}
    \caption{Compute the number of minimun spanning tree of $G$}
    \begin{algorithmic}
        \Function{NumberOfSpanningTreesOfMultigraph}{$V$, $E$}

        some black magic...

        \EndFunction

        \Function{NumberOfMST}{$V$, $E$, $w$}

        \State $W \gets \{w(e_i): e_i \in E\}$
        \State $X \gets \varnothing$
        \State $Answer \gets 1$

        \For{$w_i \in W$ in increasing order}

        \State $E_{w_i} \gets \{e_i: w(e_i) = w_i\} $
        \State $V' \gets$ connected components of $G' = (V, X)$
        \State $E' \gets \{(A, B): (x, y) \in E, x \text{ is in component } A, y \text{ is in component } B\}$
        \State $Answer \gets Answer \times$ \Call{NumberOfSpanningTreesOfMultigraph}{$V'$, $E'$}
        \State $X \gets X \cup E_{w_i}$

        \EndFor

        \Return $Answer$

        \EndFunction
    \end{algorithmic}

\end{algorithm}

The algorithm above can compute the number of MST of $G$ in polynomial time if we can get the number of spanning trees of multigraphs in polynomial time.
\begin{itemize}
    \item Effectiveness(polynomial-time):

        It is polynomial-time because the for-loop will be executed $O(n)$ times, and every statement in the loop can be done in polynomial time.

    \item Correctness:

        Consider the process of Kruskal's Algorithm to find an MST, [problem 1] shows that $T_{w_i}$ will always have the same connected components. Thus, by adding edges of same weights into that graph, the resulting connected components will always be the same. This observation indicates that there will be multiple MSTs if and only if we can add $E_{w_i}$ into the graph in different ways, and for every $w_i$, the influence on answer is independent.

        Thus at each time we add edges of same weights into the graph together, and find the number of ways these edges can connect the current graph. The number of MSTs is the product of numbers of ways for every $w_i$.
\end{itemize}

\end{document}

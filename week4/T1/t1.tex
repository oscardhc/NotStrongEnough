\usepackage{amsfonts}

\documentclass[UTF8, a4paper, linespread=1.5]{article}

\usepackage{tcolorbox, listings, algorithm, minted, algpseudocode}
\usepackage{geometry, savesym, amsmath, enumerate, indentfirst, color, amsthm, bm, extarrows, ulem}
\usepackage{amssymb}
\usepackage{nameref, hyperref}
 \geometry{top=3cm, bottom=3cm, left=1.5cm, right=1.5cm}

\usepackage{enumitem}
\setenumerate[1]{itemsep=0pt,partopsep=0pt,parsep=\parskip,topsep=5pt}
\setitemize[1]{itemsep=0pt,partopsep=0pt,parsep=\parskip,topsep=5pt}

\renewcommand\contentsname{Contents}

\tcbuselibrary{skins, breakable, theorems}

% \setlength{\leftskip}{10pt}
\setlength{\parindent}{10pt}
% \setlength{\parskip}{2em}
\renewcommand{\baselinestretch}{1.3}

\newcounter{RomanNumber}
\newcommand{\mrm}[1]{(\setcounter{RomanNumber}{#1}\Roman{RomanNumber})}

\newtcbtheorem{thm}{}
  {enhanced, theorem name and number, code={\edef\@currentlabelname{#2}}, 
  frame code={
        % \path[thick, draw] (frame.north west) -| (frame.north east) -| (frame.south east) -| (frame.south west) -| (frame.north west);
        \path[thick, draw] (frame.north west)  +(.5\baselineskip,0) -| +(0,-.5\baselineskip);
        % \path[thick, draw] (frame.north east) +(-.5\baselineskip,0) -| +(0,-.5\baselineskip);
        % \path[thick, draw] (frame.south west) +(.5\baselineskip,0) -| +(0,.5\baselineskip);
        \path[thick, draw] (frame.south east) +(-.5\baselineskip,0) -| +(0,.5\baselineskip);
    },
    left=1mm, right=1mm, top=1mm, bottom=1mm,
    colback=black!5,
    colframe=red!75!black,
    colbacktitle=black!0,
    coltitle=black!100,
    fonttitle=\bfseries}{thm}


\usepackage{environ}
\RenewEnviron{math}{%
\begin{align*}
\BODY
\end{align*}
}

\title{CS217 -- Algorithm Design and Analysis \\ Homework 7}
\date{\today}
\author{Not Strong Enough}



\begin{document}
    \maketitle

    \begin{thm}{}{}
        Let $T$ be a minimum spanning tree of $G$, and let $c\in \mathbb{R}$.
        Show that $T_c$ and $G_c$ have exactly the same connected components. (That is, two vertices $u,v\in V$ are connected in $T_c$ if and only if they are connected in $G_c$).
        You are encouraged to draw pictures to illustrate your proof.
    \end{thm}
    \begin{proof}[Solution]
        Since $T_c$ is a subset of $G_c$, if $u,v\in V$ are connected in $T_c$, then $u,v$ are connected in $G_c$.
        
        Now let's assume $u,v\in V$ are not connected in $T_c$, but are connected in $G_c$.
        We call the edge $(u,v)$ $x$, then $w(x)\leq c$.
        Also, the path in $T$ from $u$ to $v$ contains an edge whose weight is greater than c.
        We call it $y$, then $w(y)>  c$.
        If we add $x$ into $T$, this forms a cycle which contains both $x$ and $y$.
        After that, removing $y$ from $T$ still keeps the connectivity.
        We name $T'=T+x-y$.
        Now $T'$ becomes a tree, because it has $|V|-1$ edges and it's connected.
        The total weight of  $T'$ is $w(T)+w(x)-w(y)<w(T)$, which violates that $T$ is an MST\@.
        So $u,v$ are not connected in $G_c$ if they are not connected in $T_c$.

        In conclusion, two vertices $u,v\in V$ are connected in $T_c$ if and only if they are connected in $G_c$.
    \end{proof}


\end{document}


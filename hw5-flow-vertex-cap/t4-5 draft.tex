\documentclass[UTF8, a4paper, linespread=1.5]{article}

\usepackage{tcolorbox, listings, algorithm, minted, algpseudocode}
\usepackage{geometry, savesym, amsmath, enumerate, indentfirst, color, amsthm, bm, extarrows, ulem}
\usepackage{amssymb}
\usepackage{nameref, hyperref}
 \geometry{top=3cm, bottom=3cm, left=1.5cm, right=1.5cm}

\usepackage{enumitem}
\setenumerate[1]{itemsep=0pt,partopsep=0pt,parsep=\parskip,topsep=5pt}
\setitemize[1]{itemsep=0pt,partopsep=0pt,parsep=\parskip,topsep=5pt}

\renewcommand\contentsname{Contents}

\tcbuselibrary{skins, breakable, theorems}

% \setlength{\leftskip}{10pt}
\setlength{\parindent}{10pt}
% \setlength{\parskip}{2em}
\renewcommand{\baselinestretch}{1.3}

\newcounter{RomanNumber}
\newcommand{\mrm}[1]{(\setcounter{RomanNumber}{#1}\Roman{RomanNumber})}

\newtcbtheorem{thm}{}
  {enhanced, theorem name and number, code={\edef\@currentlabelname{#2}}, 
  frame code={
        % \path[thick, draw] (frame.north west) -| (frame.north east) -| (frame.south east) -| (frame.south west) -| (frame.north west);
        \path[thick, draw] (frame.north west)  +(.5\baselineskip,0) -| +(0,-.5\baselineskip);
        % \path[thick, draw] (frame.north east) +(-.5\baselineskip,0) -| +(0,-.5\baselineskip);
        % \path[thick, draw] (frame.south west) +(.5\baselineskip,0) -| +(0,.5\baselineskip);
        \path[thick, draw] (frame.south east) +(-.5\baselineskip,0) -| +(0,.5\baselineskip);
    },
    left=1mm, right=1mm, top=1mm, bottom=1mm,
    colback=black!5,
    colframe=red!75!black,
    colbacktitle=black!0,
    coltitle=black!100,
    fonttitle=\bfseries}{thm}


\usepackage{environ}
\RenewEnviron{math}{%
\begin{align*}
\BODY
\end{align*}
}

\title{CS217 -- Algorithm Design and Analysis \\ Homework 7}
\date{\today}
\author{Not Strong Enough}



\begin{document}
    \maketitle

    \begin{thm}{}{}
        Consider the induced bipartite subgraph $H_n[L_i\cup L_{i+1}]$, show that for $i<n/2$ the graph has a matching of size $|L_i|=\binom{n}{i}$
    \end{thm}
    \begin{proof}
        We do this by induction. 
        
        1. The initial step is that if $n=1,2$, the theorem is correct. 
        This is trivial since in these conditions we only need to consider $i=0$ and $|L_i|=1$. 
        Since $L_0$ and $L_1$ is connected, there is at least one edge between them. 

        2. the inductive assumption is that for $n=1,2\dots k$, the theorem is correct, 
        the inductive step proves that for $n=k+1$ it is correct. 
        In below, write each vertex $v$ in $H_{k+1}$ by its bitcode $b_v=\overline{B_1B_2\dots B_{k+1}}$, 
        where $B_i$ is the coordinate of the vertex in dimension $i$. 

        3. Now prove the theorem is true for $(n, i)$,where $n=k+1$. 
        If edge $(v,v')$ is in the matching $M$, we call $v$ and $v'$ are matched. 
        And each two matched points in $H_n$ has exactly one bit different comparing their bitcodes. 
        More generally, let $L_i$ in $H_n$ be $L_{i,n}$. 

        We form the matching by: 

        A. First, consider the collection $L_{i, k+1}^{(1)}=\left\{v\in L_{i}:b_v=\overline{b_u1}\right\}$, 
        and consider the first $k$ bits. Then the collection equals $L_{i-1, k}$. 
        It is obvious that $i-1<k/2$ and by assumption, there is a matching $M_{i-1,k}$ 
        from $L_{i-1,k}$ to $L_{i,k}$. Let the vertex $v$ in $L_{i,k+1}^{(1)}$ matches $v'$ that: 
        
        $b_v=\overline{b_t1},b_{v'}=\overline{b_{u'}1}$ where $u\in L_{i-1,k},u'\in L_{i,k}$ 
        and $u,u'$ are matched by $M_{i-1,k}$. This matching is obviously correct. 
        
        Now denote the matched vertices in $L_{i,k}$ by matching $M_{i-1,k}$ be $L_{i,k}^{(1)}$, 
        and the rest in $L_{i,k}$ be $L_{i,k}^{(2)}$. 

        B. Now consider the rest vertices in $L_{i,k+1}$(denoted by $L_{i,k+1}^{(2)}$). 
        They are in forms of $\{v:b_v=\overline{b_t0},u\in L_{i,k}\}$. Consider the subset that: 
        $\{v:b_v=\overline{b_t0},u\in L_{i,k}^{(2)}\}$. We matches these points to $L_{i+1,k+1}$ by the following rule: 
        
        $v$ is matched by $v'$ if $b_v=\overline{b_t0},b_{v'}=\overline{b_t1}$. Since $u\in L_{i,k}^{(2)}$, 
        all matched $v'$ are not ever matched before. 

        C. Now consider the last part that $\{v:b_v=\overline{b_t0},u\in L_{i,k}^{(1)}\}$. 
        
        C1. If $i<k/2$, there is a matching $M_{i,k}$ from $L_{i,k}$ to $L_{i+1,k}$. 
        Let $v$ matches $v'$ that $b_v=\overline{b_t0},b_{v'}=\overline{b_{u'}0}$ and $u,u'$ are matched by $M_{i,k}$. 

        C2. Otherwise, there is $i\geq k/2,i-1<(k+1)/2$ so there is $k=2u$. And we consider that: 
        
        (How to get this??)If $k=2i$, there is a set of disjoint paths from $L_{i-1}$ to $L_{i+1}$.(this may be done by some symmetry?) 
        If so, the set of all vertices of $L_{i,k}$ in these paths is our $L_{i,k}^{(1)}$ 
        and has a matching to $L_{i+1,k}$, adding 0 at the end of each bitcode gets what we need. 
        (just like extends $L_{i,k}^{(1)-L_{i+1,k}}$ to $L_{i,k+1}^{(1)}-L_{i+1,k+1}$).
    \end{proof} 

    My idea of exercise 5 is to prove the "how to get this" part first. 
    And then by our construction in exercise 4, choose the $L_{i-1}^{(1)}$ part to go to two side. 

\end{document}


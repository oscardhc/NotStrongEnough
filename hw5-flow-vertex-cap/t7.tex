\documentclass[UTF8, a4paper, linespread=1.5]{article}

\usepackage{tcolorbox, listings, algorithm, minted, algpseudocode}
\usepackage{geometry, savesym, amsmath, enumerate, indentfirst, color, amsthm, bm, extarrows, ulem}
\usepackage{amssymb}
\usepackage{nameref, hyperref}
 \geometry{top=3cm, bottom=3cm, left=1.5cm, right=1.5cm}

\usepackage{enumitem}
\setenumerate[1]{itemsep=0pt,partopsep=0pt,parsep=\parskip,topsep=5pt}
\setitemize[1]{itemsep=0pt,partopsep=0pt,parsep=\parskip,topsep=5pt}

\renewcommand\contentsname{Contents}

\tcbuselibrary{skins, breakable, theorems}

% \setlength{\leftskip}{10pt}
\setlength{\parindent}{10pt}
% \setlength{\parskip}{2em}
\renewcommand{\baselinestretch}{1.3}

\newcounter{RomanNumber}
\newcommand{\mrm}[1]{(\setcounter{RomanNumber}{#1}\Roman{RomanNumber})}

\newtcbtheorem{thm}{}
  {enhanced, theorem name and number, code={\edef\@currentlabelname{#2}}, 
  frame code={
        % \path[thick, draw] (frame.north west) -| (frame.north east) -| (frame.south east) -| (frame.south west) -| (frame.north west);
        \path[thick, draw] (frame.north west)  +(.5\baselineskip,0) -| +(0,-.5\baselineskip);
        % \path[thick, draw] (frame.north east) +(-.5\baselineskip,0) -| +(0,-.5\baselineskip);
        % \path[thick, draw] (frame.south west) +(.5\baselineskip,0) -| +(0,.5\baselineskip);
        \path[thick, draw] (frame.south east) +(-.5\baselineskip,0) -| +(0,.5\baselineskip);
    },
    left=1mm, right=1mm, top=1mm, bottom=1mm,
    colback=black!5,
    colframe=red!75!black,
    colbacktitle=black!0,
    coltitle=black!100,
    fonttitle=\bfseries}{thm}


\usepackage{environ}
\RenewEnviron{math}{%
\begin{align*}
\BODY
\end{align*}
}

\title{CS217 -- Algorithm Design and Analysis \\ Homework 7}
\date{\today}
\author{Not Strong Enough}



\begin{document}
    
    Obviously, this is not true for general (non-bipartite) graphs: the triangle $K_3$ has $\nu(K_3) = 1$ but it has three minimum vertex covers. The five-cycle $C_5$ has $\nu(C_5) = 2$ but has five minimum vertex covers.

    \begin{thm}{}{}
        Is there a function $f \colon \mathbf{N_0} \rightarrow \mathbf{N_0}$ such that every graph with $\nu(G) = k$ has at most $f(k)$ minimum vertex covers? How small a function $f$ can you obtain?
    \end{thm}

    \begin{proof}[Solution]
        Suppose that we have a graph $G = (V, E)$ and one of its maximum matching $M \subseteq E$ with $|M| = \nu(G) = k$.
        
        We have the two following observations:
        
        \begin{itemize}
            \item For any vertex cover $C \subseteq V$ of $G$, for every edge in $M$ there must be at least one of its endpoint which is in $C$. Otherwise there exists an edge in $M$ such that neither of its endpoints is in $C$, which means that this edge is uncovered and therefore $C$ is not a vertex cover.
            \item For any vertex $v$ which is not matched, all of its neighbors must be matched, or the edge between $v$ and one of its unmatched neighbors can be added to the maximum matching and therefore $M$ is not maximum.
        \end{itemize}
        
        We now construct a vertex set $C_0 \subseteq V$ such that for every edge $(u, v)$ in $M$, either $u \in C_0$, or $v \in C_0$, or both $u, v \in C_0$. There are $3^k$ possible $C_0$ in total.
        
        For each possible $C_0$, note that it may not be a ``vertex cover'' by far. So we try to construct another vertex set $C_1$ from $C_0$. Let $C_1$ be an empty set at the beginning. From the second observation above, for every unmatched vertex $v$, there are two cases. If all of its neighbors are in $C_0$, then we do nothing, since every edge connected to $v$ is covered by vertices in $C_0$. Otherwise we add it into $C_1$ to cover the edges which $C_0$ didn't cover. Note that $C_1$ is {\it uniquely determined by $C_0$}.
        
        Let $\mathcal{C}$ be a family of vertex covers, which is initialized to empty. Now consider $C_0 \cup C_1$. We know that it is also uniquely determined by $C_0$. If it is a vertex cover, we add it to $\mathcal{C}$. There are at most $3^k$ vertex covers in $\mathcal{C}$, since there are $3^k$ possible $C_0$, and for some $C_0$ and its corresponding $C_1$, $C_0 \cup C_1$ may not be a vertex cover.
        
        Claim that any minimum vertex cover $C$ must belong to $\mathcal{C}$. Because from the first observation, we can let the unique $C_0$ be the matched vertices covered by $C$. And then unique $C_1$ can be constructed from $C_0$. $C_0 \cup C_1$ is the minimum vertex cover when $C_0$ is fixed. So $C = C_0 \cup C_1 \in \mathcal{C}$.
        
        So there are at most $3^k$ minimum vertex covers in total, and $f(k) = 3^k$. Also note that this upper bound is {\it tight}. Just consider the triangle $K_3$ --- it has $3 = 3^1 = 3^{\nu(K_3)}$ minimum vertex covers.
    \end{proof}


\end{document}


%! Author = jinho
%! Date = 5/6/2020
\documentclass[UTF8, a4paper, linespread=1.5]{article}

\usepackage{tcolorbox, listings, algorithm, minted, algpseudocode}
\usepackage{geometry, savesym, amsmath, enumerate, indentfirst, color, amsthm, bm, extarrows, ulem}
\usepackage{amssymb}
\usepackage{nameref, hyperref}
 \geometry{top=3cm, bottom=3cm, left=1.5cm, right=1.5cm}

\usepackage{enumitem}
\setenumerate[1]{itemsep=0pt,partopsep=0pt,parsep=\parskip,topsep=5pt}
\setitemize[1]{itemsep=0pt,partopsep=0pt,parsep=\parskip,topsep=5pt}

\renewcommand\contentsname{Contents}

\tcbuselibrary{skins, breakable, theorems}

% \setlength{\leftskip}{10pt}
\setlength{\parindent}{10pt}
% \setlength{\parskip}{2em}
\renewcommand{\baselinestretch}{1.3}

\newcounter{RomanNumber}
\newcommand{\mrm}[1]{(\setcounter{RomanNumber}{#1}\Roman{RomanNumber})}

\newtcbtheorem{thm}{}
  {enhanced, theorem name and number, code={\edef\@currentlabelname{#2}}, 
  frame code={
        % \path[thick, draw] (frame.north west) -| (frame.north east) -| (frame.south east) -| (frame.south west) -| (frame.north west);
        \path[thick, draw] (frame.north west)  +(.5\baselineskip,0) -| +(0,-.5\baselineskip);
        % \path[thick, draw] (frame.north east) +(-.5\baselineskip,0) -| +(0,-.5\baselineskip);
        % \path[thick, draw] (frame.south west) +(.5\baselineskip,0) -| +(0,.5\baselineskip);
        \path[thick, draw] (frame.south east) +(-.5\baselineskip,0) -| +(0,.5\baselineskip);
    },
    left=1mm, right=1mm, top=1mm, bottom=1mm,
    colback=black!5,
    colframe=red!75!black,
    colbacktitle=black!0,
    coltitle=black!100,
    fonttitle=\bfseries}{thm}


\usepackage{environ}
\RenewEnviron{math}{%
\begin{align*}
\BODY
\end{align*}
}

\title{CS217 -- Algorithm Design and Analysis \\ Homework 7}
\date{\today}
\author{Not Strong Enough}




% Document
\begin{document}
    \maketitle
    \begin{thm}{}{}
        Let $\nu(G)$ denote the size of a maximum matching of $G=(V,E)$.
        Show that a bipartite graph $G$ has at most $2^{\nu(G)}$ mimimum vertex covers.
    \end{thm}
    \begin{proof}
        From the K\"{o}nig's Theorem, we know that the size of mimimum vertex cover is $\nu(G)$ if $G$ is a bipartite graph.
        Let $C$ be a minimum vertex cover.
        Then we can construct new minimum vertex covers by choosing vertices from $C$ and $V\setminus C$.
        In other words, all minimum vertex covers can be represented by $X\cup Y$, where $X\subseteq C, Y \subseteq V\setminus C$.
        Denote $N(A)=\{b \mid \exists a\in A, \text{ there is an edge between } a \text{ and } b \}$.
        For all $X\subseteq C$, to construct a vertex cover, $Y$ must touch all edges touched by $C\setminus X$ but not by $X$.
        \begin{itemize}
            \item If $N(C\setminus X)\cap (C\setminus X)\ne \emptyset$, there does not exist such a $Y$.
            \item  If $N(C\setminus X)\cap (C\setminus X)= \emptyset$, then $Y$ must be at least $N(C\setminus X) \cap (V\setminus C)$ to be a vertex cover.
            To make $X\cap Y$ a minimum vertex cover, $Y$ has to be $N(C\setminus X) \cap (V\setminus C)$.
        \end{itemize}

        Since $X$ has $2^{\nu(G)}$ choices, $G$ has at most $2^{\nu(G)}$ mimimum vertex covers.
    \end{proof}

\end{document}
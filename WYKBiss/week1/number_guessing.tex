\documentclass[UTF8, a4paper, linespread=1.5]{article}

\usepackage{tcolorbox, listings, algorithm, minted, algorithmic}
\usepackage{geometry, savesym, amsmath, enumerate, indentfirst, color, amsthm, bm, extarrows, ulem}
\usepackage{amssymb}
\usepackage{nameref, hyperref}
 % \geometry{top=3cm, bottom=3cm, left=1.5cm, right=1.5cm}

\usepackage{enumitem}
\setenumerate[1]{itemsep=0pt,partopsep=0pt,parsep=\parskip,topsep=5pt}
\setitemize[1]{itemsep=0pt,partopsep=0pt,parsep=\parskip,topsep=5pt}

% \usepackage{adjustbox}

\renewcommand\contentsname{Contents}

\tcbuselibrary{skins, breakable, theorems}

% \setlength{\leftskip}{10pt}
\setlength{\parindent}{10pt}
% \setlength{\parskip}{2em}
\renewcommand{\baselinestretch}{1.3}

\newcounter{RomanNumber}
\newcommand{\mrm}[1]{(\setcounter{RomanNumber}{#1}\Roman{RomanNumber})}

\newtcbtheorem{thm}{}
  {enhanced, theorem name and number, code={\edef\@currentlabelname{#2}}, 
  frame code={
        % \path[thick, draw] (frame.north west) -| (frame.north east) -| (frame.south east) -| (frame.south west) -| (frame.north west);
        \path[thick, draw] (frame.north west)  +(.5\baselineskip,0) -| +(0,-.5\baselineskip);
        % \path[thick, draw] (frame.north east) +(-.5\baselineskip,0) -| +(0,-.5\baselineskip);
        % \path[thick, draw] (frame.south west) +(.5\baselineskip,0) -| +(0,.5\baselineskip);
        \path[thick, draw] (frame.south east) +(-.5\baselineskip,0) -| +(0,.5\baselineskip);
    },
    left=1mm, right=1mm, top=1mm, bottom=1mm,
    colback=black!5,
    colframe=red!75!black,
    colbacktitle=black!0,
    coltitle=black!100,
    fonttitle=\bfseries}{thm}


\usepackage{xparse}
\NewDocumentEnvironment{qte}{m}{\begin{tcolorbox}[breakable, leftrule=2mm, rightrule=-0.1mm, toprule=-0.1mm, bottomrule=-0.1mm, arc=0mm, colframe=black!30!white, colback=white, coltext=white!50!black]}{\\\rightline{#1}\end{tcolorbox}}

\title{CS217 -- Algorithm Design and Analysis \\ Homework 1}
\date{\today}
\author{Not Strong Enough}

\begin{document}
\maketitle

\begin{thm}{}{}
A person writes two distinct integers on two cards, one per card,
and puts them on the table face down. Pick either of the two, look at it, and then guess
whether the other number is larger or smaller. Suppose that you have a good random number
generator. Prove that you have a strategy to make a correct guess with probability strictly
greater than $\frac{1}{2}$.
\end{thm}

Let the two distinct integers on cards be $p, q \in \mathbb{Z} $.

The strategy exists if we can generate a random number on $\mathbb{Z} $. Let this random number be $n$.
Our strategy is as follow: If $n > p$, we make the guess that the other number is greater.

Suppose $p < q$, there will be six situations:

\begin{table}[h!]
    \centering
    \begin{tabular}{lll}
        Seen number   & $p$     & $q$     \\
        $n < p$         & Wrong   & Correct \\
        $p < n < q$ & Correct & Correct \\
        $q < n$     & Correct & Wrong
    \end{tabular}
\end{table}

And we have the same probability to see $p$ and $q$. Thus the probability to guess correctly is
$$p = \frac{1}{2} + \frac{P(p < n < q)}{2}$$

, which is strictly greater that $\frac{1}{2}$.

For example, if we can get a random angle $\theta \in [0,2\pi)$, and generate by $n = \cot \frac{\theta}{2}$. Then $$p = \frac{1}{2} + \frac{\cot^{-1} q - \cot^{-1} p}{2\pi}.$$

And if we only have a coin, we can toss it for $k$ times to generate a random angle $k \in [0, 2\pi]$ by
$k = \frac{\text{ integer represented by the sequence }}{2^k} \cdot 2\pi$. Then apply the procedure above.
\end{document}
